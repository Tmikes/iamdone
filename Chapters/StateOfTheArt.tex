\chapter{State of the art} % Main chapter title
\label{StateOfTheArt}

\section{Volume rendering algorithms}

\section{Occlusion management strategies}

\subsection{Transfer Function}

\subsection{Segmentation}

\subsection{Deformations}

\section{Volume rendering in Virtual Reality and Augmented Reality}


Volumetric data exploration with direct volume rendering technique is of great help to visually extract relevant structures in many fields of science: medical imaging ~\cite{ljung_full_2006}, astrophysics and more recently in baggage inspection. To leverage this knowledge extraction, many techniques have been developed. 

In this section, we detail existing ones with volume visualization, transfer function, direct voxel manipulation, and focus plus context interaction.
Volume visualization can be done with geometric rendering system which transforms the data into a set of polygons representing an iso-surface. The contour tree algorithm ~\cite{carr_computing_2000} and other alternatives such as branch decomposition ~\cite{pascucci_multi-resolution_2004} are usually used to find these iso-surfaces. According to H. Guo ~\cite{guo_local_2013}, contour trees algorithms are vulnerable to noises, which can be problematic in baggage inspections since dense materials (e.g. iron) cause noises by reflecting the X-rays. For this reason we used the volume rendering technique; Chen et al. provide a review of existing techniques ~\cite{chen_3-d_2000}.
In order to investigate volumetric dataset, one can use the Transfer Function (TF). In practice, it maps the voxel density with a specific color (including its transparency). Transfer function can be 1D, 2D or nD and are of great help to isolate structures of interest in volumetric data ~\cite{kniss_multidimensional_2002}. Thanks to the color blending process, suitable transfer function can also reveal iso surface or hide density to improve the volumetric data visualization. The setup of this transfer function remains complex, but some automatic systems provide solution thanks to data analysis ~\cite{correa_size-based_2008} ~\cite{sereda_visualization_2006} ~\cite{patel_moment_2009} or user interactions ~\cite{guo_wysiwyg_2011}. Since our users have a limited knowledge on volume rendering, we developed new interaction techniques with sufficient abstraction level regarding technical constraints. These techniques will be detailed in this paper. As an example, we investigated predefined transfer functions with smooth transitions when changing their setup to reduce disruptive animation effects ~\cite{tversky_animation:_2002}.
Since graphic card power never stops to improve, new techniques allow the direct manipulation of the voxels which composes the volume to be displayed. Color tunneling introduced a set of interactions (lock, Brush, Dig) to explore 2D and 3D dataset composed of pixels of voxels ~\cite{hurter_interactive_2014} with point based rendering techniques ~\cite{sainz_point-based_2004} . Histomage also provides direct manipulation of pixels thanks to a histogram which can be used as a new selection and pixel modification tool ~\cite{chevalier_histomages:_2012}. Other interactive systems investigated pixel exploration with a lens as a focus plus context technique ~\cite{elmqvist_color_2011} ~\cite{hurter_moleview:_2011}. Our work provides additional interaction techniques with pixel based techniques ~\cite{hurter_interactive_2014} to address occlusion issues, focus and context awareness, reversibility of the actions, and continuity.  