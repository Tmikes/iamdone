% Chapter 1

\chapter{Introduction} % Main chapter title

\label{Introduction} % For referencing the chapter elsewhere, use \ref{Chapter1} 

%----------------------------------------------------------------------------------------

% Define some commands to keep the formatting separated from the content 
\newcommand{\keyword}[1]{\textbf{#1}}
\newcommand{\tabhead}[1]{\textbf{#1}}
\newcommand{\code}[1]{\texttt{#1}}
\newcommand{\file}[1]{\texttt{\bfseries#1}}
\newcommand{\option}[1]{\texttt{\itshape#1}}

%----------------------------------------------------------------------------------------


\section{Context}

Volumetric data sets are present in many fields such as medical imaging, physics, natural science, security, engineering etc. They are composed of voxels which represents the shape of these data. In fact, a volume is a scalar function of three spatial variables $(x,y,z) \in \mathbb{R}^3$. Voxel volumes can be produced by 3 main ways. The first one is to directly acquire from the real world thanks to some specific devices. As an example, X-ray scanners allow to collect data from baggage in airports. The second method is to generate the voxels through the volume by using mathematical models. For instance, a fluid flow in a basin can be represented in a volume of voxels thanks to the adequate mathematical model. The third way to produce volumetric data sets is to  rasterize vector data model using algorithms. This can be carried out for many purposes such as retrieving new insights thanks to different visualization techniques.

Volumetric data is very common nowadays. The importance of this dataset type will grow rapidly due to the development of the 3D data acquisition field, and the possibilities to perform an advanced visualization on a modern office workstation with an interactive framerate.

The dataset can be captured by various technologies, e.g. \textbf{  MRI (Magnetic resonance imaging) } , \textbf{ CT (computed tomography ) }, \textbf{PET (Positron-emission tomography ) }, \textbf{ USCT (Ultrasound computer tomography ) echolocation }. It also can be produced by physical simulations, for example, fluid dynamics or particle systems. The set of technologies mentioned before demonstrates that volumetric information plays an important role in medicine. It is used for an advanced cancer detection, visualization of aneurysms, and treatment planning. This kind of data is also very useful for non-destructive material testing via computer tomography or ultrasound. In addition, huge three-dimensional dataset is produced by geoseismic research.


An \textbf{  MRI (Magnetic resonance imaging) } is a radiology technique scan that uses magnetism, radio waves, and a computer to produce images of body structures. The MRI scanner is a tube surrounded by a giant circular magnet. The patient is placed on a movable bed that is inserted into the magnet. The magnet creates a strong magnetic field that aligns the protons of hydrogen atoms, which are then exposed to a beam of radio waves. This spins the various protons of the body, and they produce a faint signal that is detected by the receiver portion of the MRI scanner. A computer processes the receiver information, which produces an image.
MRI image and resolution is quite detailed, and it can detect tiny changes of structures within the body. For some procedures, contrast agents, such as gadolinium, are used to increase the accuracy of the images. \newline  An MRI scan can be used as an extremely accurate method of disease detection throughout the body and is most often used after the other testing fails to provide sufficient information to confirm a patient's diagnosis. In the head, trauma to the brain can be seen as bleeding or swelling. Other abnormalities often found include brain aneurysms, stroke, tumors of the brain, as well as tumors or inflammation of the spine.  \newline Neurosurgeons use an MRI scan not only in defining brain anatomy, but also in evaluating the integrity of the spinal cord after trauma. It is also used when considering problems associated with the vertebrae or inter-vertebral discs of the spine. An MRI scan can evaluate the structure of the heart and aorta, where it can detect aneurysms or tears. MRI scans are not the first line of imaging test for these issues or in cases of trauma.  \newline It provides valuable information on glands and organs within the abdomen, and accurate information about the structure of the joints, soft tissues, and bones of the body. Often, surgery can be deferred or more accurately directed after knowing the results of an MRI scan.

\textbf{  Computed tomography (CT) } is a diagnostic imaging test used to create detailed images of internal organs, bones, soft tissue and blood vessels. The cross-sectional images generated during a CT scan can be reformatted in multiple planes, and can even generate three-dimensional images which can be viewed on a computer monitor, printed on film or transferred to electronic media. CT scanning is often the best method for detecting many different cancers since the images allow to confirm the presence of a tumor and determine its size and location. CT is fast, painless, noninvasive and accurate. In emergency cases, it can reveal internal injuries and bleeding quickly enough to help save lives. \newline Computed tomography (CT) of the body uses sophisticated x-ray technology to help detect a variety of diseases and conditions. CT scanning is fast, painless, noninvasive and accurate.  Industrial CT Scanning  is a process which utilizes X-ray equipment to produce 3D representations of components both externally and internally. Industrial CT scanning has been utilized in many areas of industry for internal inspection of components. Some of the key uses for CT scanning have been flaw detection, failure analysis, meteorology, assembly analysis, image-based finite element methods and reverse engineering applications. CT scanning is also employed in the imaging and conservation of museum artifacts.


 \textbf{PET (Positron-emission tomography ) } uses small amounts of radioactive materials called radio-tracers, a special camera and a computer to help evaluate your organ and tissue functions. By identifying body changes at the cellular level, PET may detect the early onset of disease before it is evident on other imaging tests. It is a type of nuclear medicine imaging. Nuclear medicine is a branch of medical imaging that uses small amounts of radioactive material to diagnose and determine the severity of or treat a variety of diseases, including many types of cancers, heart disease, gastrointestinal, endocrine, neurological disorders and other abnormalities within the body. Because nuclear medicine procedures are able to pinpoint molecular activity within the body, they offer the potential to identify disease in its earliest stages as well as a patient's immediate response to therapeutic interventions. Nuclear medicine images can be superimposed with computed tomography (CT) or magnetic resonance imaging (MRI) to produce special views, a practice known as image fusion or co-registration. These views allow the information from two different exams to be correlated and interpreted on one image, leading to more precise information and accurate diagnoses.


\textbf{ USCT (Ultrasound computer tomography ) echolocation }  uses ultrasound waves for creating images. In the first measurement step a defined ultrasound wave is generated with  ultrasound transducers, transmitted in direction of the measurement object and received with other or the same ultrasound transducers. While traversing and interacting with the object the ultrasound wave is changed by the object and carries now information about the object. After being recorded the information from the modulated waves can be extracted and used to create an image of the object in a second step. Unlike X-ray or other physical properties which provide typically only one information, ultrasound provides multiple information of the object for imaging: the attenuation the wave's sound pressure experiences indicate on the object's attenuation coefficient, the time-of-flight of the wave gives the speed of sound information, and the scattered wave indicates on the echogenicity of the object (e.g. refraction index, surface morphology, etc.)


To visualize these type of data-sets, different rendering algorithms can be used. There are two major approaches to volume rendering. The first approach is to use ray casting based algorithms. They directly come from the rendering equation. They consists in shooting rays for each pixel of the final rasterized 2D image, sampling along the part of the ray located inside the volume, shading the sampling points, and compositing all the sampling points.   
The second approach to render volumetric data sets is to use plane compositing. It consists in accumulating information over the whole view plane for each plane of voxels in the data set. When each plane is processed, a pixel of the final rasterized 2D image is updated. This technique is texture-based and uses slices of the 3D Volume. Theses slices can be either aligned with the data set or with the viewing plane.

During this thesis, we mainly focused on two main topics: 

\begin{itemize}

\item A user study where we investigated the specific activity of baggage inspection, and proposed an interactive visualization system to support their volumetric data exploration according to the requirements and constraints of this field. 

\item Focus+Context techniques for occlusion management in volumetric data visualization.

\end{itemize}


 
 \section{Problem}
 
  Interacting with volumetric data sets is not trivial. In fact, 3D volume visualization face many challenges such as the occlusion management and the computational time.
 In volume rendering, occlusion management is a challenge. As such, in 3d representations of volumes, some areas or objects (subsets) can be partially or fully hidden by others because of their locations. Transfer functions are used to match the volumetric data to colors in a meaningful way. Therefore, they are a good way to reduce occlusion and make visible interesting features.  However, it is still difficult to create a good transfer function especially when the data are heterogeneous. In fact, designing a good transfer function depends heavily on the type of dataset and on the user's purpose. For instance, in the field of baggage inspection, the variation of densities prevents to create a unique transfer function for each baggage. In contrast, it is easier to design a good transfer function for a system dedicated to visualizing the same type of datasets (brain CT scans, bone tissues, etc.). 
 
 \subsection{Baggage inspection}
 
  Since volumetric data-sets are more and more used in many areas thanks to technological breakthroughs, switching from the old systems working with 2D images to the newest ones with 3D is not straightforward and easy.  In the field of baggage inspection, the displayed 2D scanned image can suffer from four issues or dissimulation strategy.

\textbf{Superposition}: A threat (e.g. prohibited object like knife, cutter…) may be sheltered behind dense materials. Sometimes, it is possible to see through these blind shield using some functionalities such as high penetration (enhanced X-ray power) or image processing (contrast improvement). 

\textbf{Location}: Depending on its location inside the luggage, a threat can be difficult to detect. Objects located in the corners, in the edges or inside the luggage’s frame are very difficult to identify.

\textbf{Dissociation}: Another way to dissimulate a threat is to separate and to spread parts of it in the luggage (weapon or explosive are composed of many separated items like the trigger, the cannon...). This dissociation can be combined with other dissimulation techniques.

\textbf{Lure}: An ill-intentioned individual may use a lure to hide the real threat. For instance, a minor threat like a small scissors may be clearly visible and catch security agent's attention while a more important threat remains hidden.


3D baggage scan exploration are one potential solution of such limitations. Few systems investigated this activity domain with interactive volumetric exploration tools like \cite{Li:2012:LVV:2425296.2425325}. Even if extensive works have been done in medical 3D scan exploration and manipulation by \cite{preim2013visual}, there is a great opportunity to adapt and develop new interaction and data manipulation techniques to support 3D baggage exploration.
 
 
 \subsection{Occlusion management using Focus+Context techniques}
 
 Direct volume rendering (DVR) is a pervasive visualization technique for displaying 3D scalar fields with applications in engineering, material sciences, and medical imaging sciences. However widely adopted, and able to handle large datasets at interactive rates, DVR inherently suffers from the problem of \emph{occlusion}: Structures of interest located deep in the volume, called next \emph{targets}, can be hard to spot and/or explore.


To address this issue, various techniques have been designed including transfer functions, segmentation, selection, and clipping. Yet, all such techniques have limitations.  \emph{Global} mechanisms, like transfer function editing, can remove both occluders and targets if these have similar densities. In certain applications, carefully designed transfer functions exist and should be used without (significant) modifications to facilitate understanding and user training\,\cite{4276082}. \emph{Local} mechanisms like segmentation, selection, or clipping are more effective in manipulating data confined to a given spatial region. Yet, many such mechanisms assume that one can easily and accurately select targets to remove them (occluders) or keep them (occluded). This is hard to do when \emph{e.g.} one does not have direct access to the targets, or when significant 3D interaction is required to select occluder(s).


A different way to handle occlusion is to use \emph{lenses}. These are flexible lightweight tools which enable local and temporary modifications of the DVR to reveal targets while keeping the global visualization context\,\cite{595268,CGF:CGF12871,6327262}. However, efficiently selecting the target and  removing all in-between occluders is still challenging. More specifically, most existing occlusion management techniques do not simultaneously meet all following requirements: 


\begin{itemize}

\item rapidly create an unobstructed view of the target (R1),

\item allow a flexible local exploration of the target zone (R2),

\item keep the context in which the target is visually embedded (R3),

\item handle datasets where the target and occluders cannot 
be separated by transfer function manipulations (R4).

\end{itemize}

\section{Research Questions}



\section{ Thesis outline }


