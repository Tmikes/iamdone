% Chapter 1

\chapter{Introduction} % Main chapter title

\label{Introduction} % For referencing the chapter elsewhere, use \ref{Chapter1} 

%----------------------------------------------------------------------------------------

% Define some commands to keep the formatting separated from the content 
\newcommand{\keyword}[1]{\textbf{#1}}
\newcommand{\tabhead}[1]{\textbf{#1}}
\newcommand{\code}[1]{\texttt{#1}}
\newcommand{\file}[1]{\texttt{\bfseries#1}}
\newcommand{\option}[1]{\texttt{\itshape#1}}

%----------------------------------------------------------------------------------------


Volumetric data sets are present in many fields such as medical imaging, physics, natural science, security, engineering etc. They are composed of voxels which represents the shape of these data. In fact, a volume is a scalar function of three spatial variables $(x,y,z) \in \mathbb{R}^3$. Voxel volumes can be produced by 3 main ways. The first one is to directly acquire from the real world thanks to some specific devices. As an example, X-ray scanners allow to collect data from baggage in airports. The second method is to generate the voxels through the volume by using mathematical models. For instance, a fluid flow in a basin can be represented in a volume of voxels thanks to the adequate mathematical model. The third way to produce volumetric data sets is to  rasterize vector data model using algorithms. This can be carried out for many purposes such as retrieving new insights thanks to different visualization techniques.

To visualize these type of data sets, different rendering algorithms can be used. There are two major approaches to volume rendering. The first approach is to use ray casting based algorithms. They directly come from the rendering equation. They consists in shooting rays for each pixel of the final rasterized 2D image, sampling along the part of the ray located inside the volume, shading the sampling points, and compositing all the sampling points.   
The second approach to render volumetric data sets is to use plane compositing. It consists in accumulating information over the whole view plane for each plane of voxels in the data set. When each plane is processed, a pixel of the final rasterized 2D image is updated. This technique is texture-based and uses slices of the 3D Volume. Theses slices can be either aligned with the data set or with the viewing plane.

However, interacting with such data sets is not trivial. In fact, 3D volume visualization face many challenges such as the occlusion management and the computational time.
 In volume rendering, occlusion management is a challenge. As such, in 3d representations of volumes, some areas or objects (subsets) can be partially or fully hidden by others because of their locations. Transfer functions are used to match the volumetric data to colors in a meaningful way. Therefore, they are a good way to reduce occlusion and make visible interesting features.  However, it is still difficult to create a good transfer function especially when the data are heterogeneous. In fact, designing a good transfer function depends heavily on the type of dataset and on the user's purpose. For instance, in the field of baggage inspection, the variation of densities prevents to create a unique transfer function for each baggage. In contrast, it is easier to design a good transfer function for a system dedicated to visualizing the same type of datasets (brain CT scans, bone tissues, etc.).